\section{\ocaml Implementation}
\label{sec:ocaml}

\lstset{language=[Objective]Caml}

We also implemented our
language generation algorithm in \ocaml.
% The \ocaml version only implements the ``latest'' version of the
% algorithm with a segmented representation, fast backward lookup and convolutions
% for concatenation and star, and symbolic representation of segments.
% One of the goal of this implementation was to test
% pure \ocaml regular expression libraries such as ocaml-re\footnote{\url{https://github.com/ocaml/ocaml-re}} and
% mulet\footnote{\url{https://github.com/let-def/mulet}}. 
The main goal of this implementation is to experiment with strictness
and various data structures for segments. 
The key idea is that the internal order on words in a segment does not matter
because each segment only contains words of the same length.
All we need is a data structure that supports the stream
operations.
%
To facilitate such experimentation, we implemented the
algorithm as a functor whose signature is shown in
\cref{code:sigs:segment,code:sigs:word,code:sigs:regen}.
\ocaml{} functors are parameterized modules that take modules
as argument and return modules. Our implementation
takes two data structures as arguments, words and segments,
to test different representations without changing
the code.
% It also forced us to find the minimal set of operations
% needed to implement the algorithm.

\begin{figure}[tp]
\begin{lstlisting}
module type WORD = sig
  type char
  type t
  val empty : t
  val singleton : char -> t
  val append : t -> t -> t
end
\end{lstlisting}
\vspace{-\baselineskip}
    \caption{Operations on words}
    \label{code:sigs:word}

\begin{lstlisting}
module type SEGMENT = sig
  type elt (* Elements *)
  type t (* Segments *)

  val empty: t
  val is_empty: t -> bool
  val singleton: elt -> t

  (* Set operations *)
  val union: t -> t -> t
  val inter: t -> t -> t
  val diff: t -> t -> t
  val append: t -> t -> t
  val merge: t list -> t

  (* Import/Export *)
  val of_list: elt list -> t
  val iter: t -> (elt -> unit) -> unit

  (** For transient data-structures *)
  val memoize: t -> t
end
\end{lstlisting}
\vspace{-\baselineskip}
    \caption{Operations on segments}
    \label{code:sigs:segment}
    
\begin{lstlisting}
module Regenerate
    (Word : WORD)
    (Segment : Segments.S with type elt = Word.t)
: sig
  type lang
  val gen : Segment.t -> Word.char regex -> lang
  val iter : lang -> (Word.t -> unit) -> unit
end
\end{lstlisting}
\vspace{-\baselineskip}
    \caption{Language generation as a functor}
    \label{code:sigs:regen}
\end{figure}

\paragraph{Characters and Words}

\autoref{code:sigs:word} contains the signature for words.
It provides
the empty word (for \code{One}),
singleton words (for \code{Atom}), and append.
Neither an ordering nor a length operation is needed:
Comparison is encapsulated in the segment
data structure and the length of a word is the index of the segment in
which it appears.
%
This signature is satisfied by the \ocaml \code{string}
type (\ie arrays of bytes), arrays, lists of characters, or ropes. The
type of individual characters is unrestricted.

\paragraph{Segments}

\autoref{code:sigs:segment} contains the signature for segments.
% The first group of operations creates and tests for empty segments and
% singleton segments. 
The main requirement is to support the operations on power series as described in \autoref{sec:gener-cross-sect} and the set operations
\code{union}, \code{inter} and \code{inter}.
%
The product described in \autoref{eq:1} is decomposed in two parts:
\begin{itemize}[leftmargin=*]
\item An \code{append} function to implement $U_i V_{n-i}$. It computes the
  product of two segments by appending their elements.
\item A \code{merge} operation which computes the union of an arbitrary number
  of segments. It collects the segments obtained
  by invocations of \code{append}.
\end{itemize}
%
Experimentation with transient data-structures requires
an explicit \code{memoize} function that avoids recomputing segments accessed
multiple times. 
%
Finally, the functions  \code{of_list} and \code{iter} import and
export elements to and from a segment.

\subsection{Core Algorithm}

The core algorithm follows the Haskell version. The power series
is implemented using a thunk list in the style of \citet{DBLP:conf/cpp/Pottier17}
with some special-purpose additions:

\begin{lstlisting}
type node =
  | Nothing
  | Everything
  | Cons of Segment.t * lang
and lang = unit -> node
\end{lstlisting}

An enumeration is represented by a function which takes a unit argument and returns
a node. A node, in turn, is either \code{Nothing} or a \code{Cons} of an
element and the tail of the sequence.
% The empty enumeration, for instance, is
% represented as \code{fun () -> Nothing}.
% This representation is lazy, fast, lightweight, and almost as easy to
% manipulate as regular lists.
% \footnote{See
%   \url{https://github.com/ocaml/ocaml/pull/1002} for a long discussion
%   on the topic.}
The additional constructor \code{Everything} allow to manipulate
full languages symbolically, as in \cref{sec:more-finite-repr}.

% The rest of the implementation is similar to the \haskell one.
As an example, the implementation of language union is shown below.
The trailing unit argument \code{()} drive the evaluation of the
sequence lazily. With this definition, \code{union s1 s2} cause no
evaluation before it is applied to \code{()}.
\begin{lstlisting}
let rec union s1 s2 () = match s1(), s2() with
  | Nothing, x | x, Nothing -> x
  | Everything, _ | _, Everything -> Everything
  | Cons (x1, next1), Cons (x2, next2) -> 
    Cons (Segment.union x1 x2, union next1 next2)
\end{lstlisting}

% The concatenation of languages demonstrates how the main algorithm can
% be expressed once the segment operations have been abstracted:.
% To build $U \cdot V$, we first build the $n$th term
% $(U \cdot V)_n = \bigcup_{i=0}^n U_i V_{n-i}$.
% We use both \code{Segment.append} to implement the product
% of segments and the concatenation of words and \code{Segment.merge} to merge
% all the resulting segments.
% \begin{lstlisting}
% let term_of_length map1 map2 n =
%   let combine_segments i =
%     Segment.append (IntMap.find i map1) (IntMap.find (n - i) map2)
%   in
%   List.(range 0 n) |> List.rev_map combine_segments |> Segment.merge
% \end{lstlisting}

% We then collect all the terms by synchronized recursion over the power series $U$
% and $V$:
% \begin{lstlisting}
% let rec collect n map1 map2 seq1 seq2 () = match seq1 (), seq2 () with
%   | Cons (segm1, seq1), Cons (segm2, seq2) ->
%     let map1 = IntMap.add n (Segment.memoize segm1) map1 in 
%     let map2 = IntMap.add n (Segment.memoize segm2) map2 in
%     Cons (term_of_length map1 map2 n, collect (n+1) map1 map2 seq1 seq2)
% \end{lstlisting}

% The \code{IntMap} module provides a functional implementation of maps
% keyed by integers. The maps are used to quickly access
% segments of smaller index that have been computed in earlier invocations of
% \code{collect}. As such segments are accessed
% multiple times, we use \code{memoize} to avoid computing them over and
% over again.
% Functional maps are sufficient because the size of a map is equal
% to the maximum word length, which does not get excessively large.

% Finally, we initialize \code{collect} with empty maps.
% \begin{lstlisting}[numbers=none]
% let concatenate = collect 0 IntMap.empty IntMap.empty
% \end{lstlisting}


% The main difference compared to the \haskell version relates to the use
% of recursion to define \code{star}-related operations.
% For instance, the \haskell implementation defines \code{lsigmastar} as
% \code{[T.empty] : map extend lsigmastar}. Such a recursive definition
% can not be translated directly to \ocaml, since it is not compatible with
% strict-by-default evaluation and will be prevented by the compiler.
% Instead, we simply use an accumulator to keep track of the latest version
% of the language.
% A similar change was made for the definition of \code{star}.
% \begin{lstlisting}
% let sigma_star sigma =
%   let rec collect acc () =
%     Iter.Cons 
%       (acc, collect (Segment.append sigma acc))
%   in
%   collect (Segment.return W.empty)
% \end{lstlisting}


\subsection{Data Structures}

Our parameterized implementation enables experimentation with various
data structures for segments. We present several possibilities before
comparing their performance.

\paragraph{Ordered enumerations}

Ordered enumerations, represented by thunk-lists, make
for a light-weight set representation.
To use an order, we require \code{compare} and
\code{append} on words.  The \code{OrderedMonoid} signature
captures these requirements. \autoref{code:thunklist} shows the
resulting functor \code{ThunkList}.

% Most of the implementation is straightforward. For example, here is the
% \code{append} function, where \code{>>=} is bind (or \code{concatMap})
% and \code{>|=} is map.

% \begin{lstlisting}
% let append l1 l2 =
%   l1 >>= fun x -> l2 >|= fun y -> Elt.append x y
% \end{lstlisting}

The n-way merge, 
was implemented using a priority heap which holds pairs composed
of the head of an enumeration and its tail. When a new element is required in the
merged enumeration, we pop the top element of the heap, deconstruct
the tail and insert it back in the heap.

% \begin{lstlisting}
% let merge l =
%   let cmp (v1,_) (v2,_) = K.compare v1 v2 in
%   let merge (x1, s1) (_, s2) = (x1, s1@s2) in
%   let push h s =
%     match s() with Nil -> h | Cons (x, s') -> Heap.insert h (x, [s'])
%   in
%   let h0 = List.fold_left push (Heap.empty ~cmp ~merge) l in
%   let rec next heap () =
%     if Heap.is_empty heap then Nil else begin
%       let (x, seq), heaps = Heap.pop heap in
%       let new_heap = List.fold_left push heaps seq in
%       Cons (x, next new_heap)
%     end
%   in
%   next h0
% \end{lstlisting}

\begin{figure}[tp]
  \centering
\begin{lstlisting}
module type OrderedMonoid = sig
  type t
  val compare : t -> t -> int
  val append : t -> t -> t
end
module ThunkList (Elt : OrderedMonoid) :
  SEGMENTS with type elt = Elt.t
\end{lstlisting}
  \vspace{-\baselineskip}
  \caption{Signature for \texttt{ThunkList}}
  \label{code:thunklist}
\end{figure}

\paragraph{Transience and Memoization}

During concatenation and star, we iterate over segments multiple times.
As thunk lists are transient, iterating multiple times over the same list
will compute it multiple times. To avoid this recomputation, we can implement memoization
over thunk lists by using a growing vector as cache.
% pushing the elements in a growing vector as they are
% computed. Before evaluating a new thunk, we first check if it is already available
% in the vector. Otherwise, evaluate it, push it into the vector and return it.
% \begin{lstlisting}
% let memoize f =
%   let r = CCVector.create () in
%   let rec f' i seq () =
%     if i < CCVector.length r
%     then CCVector.get r i
%     else 
%       let l = match seq() with
%         | Nil -> Nil
%         | Cons (x, tail) -> Cons (x, f' (i+1) tail)
%       in
%       CCVector.push r l;
%       l
%   in
%   f' 0 f
% \end{lstlisting}
%
Such a memoization function incurs a linear cost on enumerations. To test
if this operation is worthwhile we implemented two modules:
\code{ThunkList} without memoization and \code{ThunkListMemo}
with the implementation described above.

\paragraph{Lazy Lists}

\ocaml also supports regular lazy lists using the builtin \code{Lazy.t} type.
%
% \begin{lstlisting}
% type 'a node =
%   | Nil
%   | Cons of 'a * 'a lazylist
% type 'a lazylist = 'a node Lazy.t
% \end{lstlisting}
%
We implemented a \code{LazyList} functor which is identical to
\code{ThunkList} but uses lazy lists.

\paragraph{Strict Sets}

As the main operations on segments are set operations, one might 
expect a set implementation to perform well. We implemented segments as sets
of words using \ocaml's built-in \code{Set} module which relies on
balanced binary trees.
The only operations not implemented by \ocaml's standard library are
the n-way merge and the product.
%, which can be implemented using folds and unions.

\paragraph{Tries}

Tries \cite{Fredkin1960} are prefix trees where each branch is labeled
with a character and each node may contain a value. Tries are commonly used
as maps from words to values where a word belongs to its domain if there is a
path reaching a value labeled with the characters in the word.
Tries seem well adapted to our problem:
since all words in a segment have the same length, we only need values at the leaves.
%   % As no prefixes need to be represented.
% \item The \code{append} operation on tries can be implemented by
%   grafting the second trie to all the leaves of the first one.
% \end{itemize}
%
Hence, we can implement tries like tries of integers \cite{Okasaki98fastmergeable}.
For simplicity, we do not use path compression, which means
that branches are always labeled with one character.
A trie is either \code{Empty}, a \code{Leaf} or a \code{Node} containing a map from characters
to its child tries.
% As we are only interested in the paths, we consider tries
% without values. 
%
% The implementation of most operations follows the literature.
The only novel operation is \code{append} which computes the product of two sets.
It can be implemented in a single traversal which grafts the
appended trie \code{t0} at each leaf of \code{t}, without copies.

\begin{lstlisting}
type trie = Empty | Leaf | Node of trie CharMap.t
let rec append t t0 = match t with
  | Empty -> Empty | Leaf -> t0
  | Node map -> 
    CharMap.map (fun t' -> append t' t0) map
\end{lstlisting}

%%% Local Variables:
%%% mode: latex
%%% TeX-master: "main"
%%% End:
