\section{Improvements}
\label{sec:improvements}

\subsection{Segmented representation}
\label{sec:segm-repr}

Two operations on the LLO representation, \code{concatenate} and
\code{star}, internally transform their inputs into segment
sequences. Such a transformation forth an back seems wasteful, so we
stipulate to perform the entire generation in terms of segment
sequences. The transformation to a language only happens as the final
step.

As a second refinement, we try to address the productivity concern
discussed in Section~\ref{sec:motivation-discussion}. The approach is
to let finite segment sequences represent finite languages. Hence,
\begin{lstlisting}
  zero = []
  one = [[T.empty]]
  atom t = [[], [T.singleton t]]
\end{lstlisting}

We discuss the remaining issues with the code for concatenation.
\begin{lstlisting}
  concatenate lx ly = collect 0
    where
      collect n =
        (foldr ILO.union [] $ map (combine n) [0 .. n]) : collect (n+1)
      combine n i =
        [T.append x y | x <- lx !!! i, y <- ly !!! (n - i)]
\end{lstlisting}
The code relies on a special indexing operation that returns an empty
list if indexing occurs beyond the end of a segment list:
\begin{lstlisting}
(!!!) :: [[a]] -> Int -> [a]
[]       !!! n = []
(xs:xss) !!! 0 = xs
(xs:xss) !!! n = xss !!! (n - 1)
\end{lstlisting}
But the concatenation of finite languages still yields a partial list.

\clearpage{}
%%% Local Variables:
%%% mode: latex
%%% TeX-master: "main"
%%% End:
