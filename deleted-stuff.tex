


 contains an implementation
of a finite version of the generator module according to the preceding
discussion. The implementation of \texttt{star} follows the definition
of $U^*$ literally. It first recursively creates a list \texttt{lxs}
of the iterates $U_\bullet^i$ where
$U_\bullet = U \setminus \{\varepsilon\}$, concatenates all of
them\footnote{It is easy to see that $U^* = U_\bullet^*$.}, removes
the duplicates, and imposes the limit. Removing duplicates introduces
quadratic worst-case time complexity in the size of the output.

The implementation of \texttt{complement} generates the language
$\Sigma^*$ analogously to the construction of $U^*$ in
\texttt{star}, uses the list difference operator
\texttt{L.\textbackslash\textbackslash} to remove elements of
\texttt{lx}, and finally imposes the limit. Its worst-case run time is
$O(m\cdot n)$ where $m$ is the limit and $n = \code{length lx}$.

In summary, the naive approach in
Figure~\ref{fig:regular-operators-0} can generate infinite languages,
but has many drawbacks that lead to duplication and incompleteness
(words in the language are not enumerated). Moreover, the complement
is not computable for this approach.

The finite approach in Figure~\ref{fig:finite-regular-operators}
imposes an arbitrary limit on the number of generated words. This
limit can lead to omitting words in nonempty languages where $P (w\in
R) = 0$. Moreover, there are many places (in \texttt{union},
\texttt{concatenate}, and \texttt{star}) with quadratic worst-case
time complexity. 

At this point, the question is: Can we do better? Can we come up with
a generator that supports finite as well as infinite languages
efficiently without incurring extraneous quadratic behavior?


\subsection{Ordered Enumeration}
\label{sec:ordered-enumeration}
\begin{figure}[tp]
\begin{lstlisting}
union :: (Ord t) => [t] -> [t] -> [t]
union xs@(x:xs') ys@(y:ys') =
  case compare x y of
    EQ -> x : union xs' ys'
    LT -> x : union xs' ys
    GT -> y : union xs ys'
union xs ys = xs ++ ys
\end{lstlisting}

\begin{lstlisting}
intersect :: (Ord t) => [t] -> [t] -> [t]
intersect xs@(x:xs') ys@(y:ys') =
  case compare x y of
    EQ -> x : intersect xs' ys'
    LT -> intersect xs' ys
    GT -> intersect xs ys'
intersect xs ys = []
\end{lstlisting}

\begin{lstlisting}
difference :: (Ord t) => [t] -> [t] -> [t]
difference xs@(x:xs') ys@(y:ys') =
  case compare x y of
    EQ -> difference xs' ys'
    LT -> x : difference xs' ys
    GT -> difference xs ys'
difference xs ys = xs
\end{lstlisting}
  \caption{Union, intersection, and difference by merging lists}
  \label{fig:merging-lists}
\end{figure}

First, we concentrate on improving on the quadratic behavior. The key
to improve the complexity of union, intersection, and complement lies
in representing a language by a strictly increasingly sorted list.
In this case, the three operations can be implemented by variations of
the merge operation on lists as shown in
Figure~\ref{fig:merging-lists}.

The merge-based operations run in linear time on finite
lists. However, the operations in Figure~\ref{fig:merging-lists} are
incomplete on infinite lists. As 
an example of the incompleteness, consider the languages $U = a\cdot
(a+b)^*$ and the singleton language $V = \{b\}$ where $\Sigma = \{a, b\}$
with $a<b$, represented as strictly increasing lists, the infinite
list \code{lu} and the singleton list \code{lv}. The problem is that
the list \code{union lu lv} does not contain the word $b$; more
precisely, \code{T.singleton 'b' `elem` union lu lv} does not
terminate whereas \code{u `elem` union lu lv} yields \code{True} for
all \code{u} in \code{lu}. The source 
of the problem is that Haskell's standard ordering of lists and
\code{Text} is the \emph{lexicographic ordering}, which we call
$\le$. It relies on an underlying total ordering on $\Sigma$ and is
defined inductively:
\begin{mathpar}
  \inferrule{}{\varepsilon  \le v}
  
  \inferrule{u \le v}{au \le av}

  \inferrule{a < b}{au \le bv}
\end{mathpar}

This total ordering
is often used for $\Sigma^*$, but it has the property that
there are words $v<w$ such that there are infinitely many words $u$
with $v<u$ and $u<w$. For example, $v=a$, $w=b$, and $u\in U \setminus
\{a\}$, which explains the nonterminating behavior just exhibited.

Fortunately, there is another total ordering on words with better
properties. The
\emph{length-lexicographic ordering} 
For the sake of simplicity, we assume from now on that \code{T.Text}
is ordered by \code{llocompare} and call the representation of a
language by a strictly increasing list in length-lexicographic order
its \emph{LLO representation}. 

With the LLO representation, the operations \code{union},
\code{intersect}, and \code{difference} run in linear time. If the input languages
are finite of size $m$ and $n$, respectively, then $O(m+n)$
comparisons, pattern matches, and cons operations are needed.
Moreover, the operations are complete in the sense that any element in
the output is sure to be detected by a terminating computation. 
It is easy to implement a version of \code{elem} that exploits the LLO ordering,
such that the element test is decidable for any infinite
language.



\subsubsection{Concatenation}
To implement concatenation, we are in the following situation. Given
two languages $U, V \subseteq \Sigma^*$ in LLO representation,
produce the LLO representation of $U \cdot V =  \{ u\cdot v \mid u\in
U, v\in V\}$. If we compute the product naively as in
Figure~\ref{fig:regular-operators-0}, then the output is not in LLO
form:\footnote{The example relies on the operator
  \lstinline{Data.Monoid.<>} to append strings in any representation.}  
\begin{verbatim}
λ> let lu = ["a", "ab"]
λ> let lv = ["", "b", "bb"]
λ> [ u<>v | u <- lu, v <- lv ]
["a","ab","abb","ab","abb","abbb"]
\end{verbatim}
In fact, the output violates both constraints: it is not increasing
and it has duplicates.

Perhaps the following observation helps: for each $u\in U$, the LLO
representation of the language $u\cdot V$ can be trivially produced
because the list \lstinline{[ u<>v |  v <- lv ]} is strictly
increasing. This observation motivates the following definition of
language concatenation (using \code{union} from Figure~\ref{fig:merging-lists}).
\begin{lstlisting}
concatenate' :: Lang -> Lang -> Lang
concatenate' lx ly =
  foldr union [] $ [[ T.append x y | y <- ly] | x <- lx]
\end{lstlisting}
This definition works fine as long as \code{lx} is finite. If it is
infinite, then the \code{foldr} creates an infinitely deep nest of
invocations of \code{union}, which do not make progress because
\code{union} is strict in both arguments.

At this point, the theory of formal languages can help. The notion of
a \emph{formal power series} has been invented to reason about and
compute with entire languages \cite{DBLP:books/daglib/0067812,DBLP:books/sp/KuichS86}. In full
generality, a formal power series is a mapping from $\Sigma^*$ into a
semiring $S$ and we write $\FPS{S}$ for the set of these
mappings. Formally, an element $r \in \FPS{S}$ is written as the
formal sum
\begin{align*}
  r &= \sum_{w \in \Sigma^*} (r,w) \cdot w
\end{align*}
where $(r,w) \in S$ is the coefficient of $w$ in $r$.
A popular candidate for this semiring is the boolean semiring $B$
because $\FPS{B}$ is isomorphic to the set of languages over
$\Sigma$. This isomorphism maps $L\subseteq\Sigma^*$ to its
characteristic series $r_L$ where $(r_L, w) = (w \in L)$.

The usual language operations have their counterparts on formal power
series. We consider just three of them where the ``additions'' and ``multiplications'' on the
right side of the definitions take place in the underlying semiring.
\begin{align*}
  &\text{Sum:}
  & (r_1 + r_2, w) &= (r_1, w) + (r_2, w) \\
  &\text{Hadamard product:}
  &(r_1 \odot r_2, w) &= (r_1, w) (r_2, w) \\
  &\text{Product:}
  & (r_1 \cdot r_2, w) &= \sum_{u\cdot v=w} (r_1, u) (r_2,v) 
\end{align*}

Ok, so what is the connection between formal power series and the LLO
representation? The LLO representation of a language $L$ can be viewed as a systematic
enumeration of the indexes of the non-zero coefficients of the
characteristic power series of $L$. Thus, the \code{union} operator
corresponds to the sum and the \code{intersect} operator corresponds
to the Hadamard product (in $B$ the $+$ and $\cdot$ operators are
$\vee$ and $\wedge$). 

We also get a hint for computing the product. To find out whether $w
\in U \cdot V$ we need to find $u\in U$ and $v\in V$ such that $u\cdot
v = w$. Abstracting from this observation, we obtain that building a
word $w$ with $|w| = n$ requires two words $u\in U$ and $v\in V$ such that $|u| +
|v| = |w| = n$. Hence, it would be useful to have a representation
that exposes the lengths of words.

To obtain such a representation, we represent a language as a power series
\begin{gather*}
  L = \sum_{n=0}^\infty L_nx^n
\end{gather*}
where, for all $n$, $L_n \subseteq \Sigma^n$, a decomposition which corresponds
directly to the definition of $\Sigma^*$.
