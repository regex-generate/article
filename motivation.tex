\section{Motivation}
\label{sec:motivation}

Let's say someone implemented a clever algorithm, say \texttt{match}, for
regular expression matching. 
We do not trust this implementation and would like to test it.
Testing the implementation requires us to come up with test cases and
implement a test oracle, say \texttt{matchOracle}.

A test case consists of a regular expression \texttt{r} and an input
string \texttt{s}. Running the test means to execute \texttt{match r
  s} and check whether the result is correct by comparing with
\texttt{matchOracle r s}. But how do we know whether
\texttt{matchOracle} is correct, short of verifying it?

A popular way of conducting this test is using QuickCheck
\cite{quickcheck}, which performs property-based random testing. Using
QuickCheck, we would write a generator for regular expressions and then
use the generator for strings to generate many inputs for a
generated regular expression.

However, this approach has a
catch. Depending on the language of the regular expression, the
probability that a random string is a member of the language can be
severly skewed. As an example, consider the language $L = (ab)^*$ over the
alphabet $\Sigma = \{a, b\}$. Although $L$ contains infinitely many
words, the probability that a random word of
length $n$ is an element of $L$ is
\begin{itemize}
\item $0$ if $n$ is odd and
\item $\frac{1}{2^n}$ if $n$ is even.
\end{itemize}
Thus, the probability $p_n$ that a random word of length less than or equal to
$n$ is an element of $L$ is way smaller:
\begin{align*}
  p_n &= \frac{\lfloor n/2 \rfloor}{2^{n+1} - 1}
        \le \frac{n}{2^{n+2} - 2}
\end{align*}
Hence, the probability of (uniformly) randomly
selecting a word in $L$ is zero in the limit.

Wouldn't it be nice to have a systematic and obviously correct means
of generating words \textbf{inside} of $L$ and \textbf{outside} of
$L$? Such a generation algorithm would obviate the need for an oracle
and it would make sure that we can control the number of test inputs
in the language and in the language's complement.

\subsection{Research Question}
\label{sec:research-question}

\begin{figure}[tp]
  \begin{align*}
    r, s &::= 
  \end{align*}
  \caption{Generalized regular expressions}
  \label{fig:generalized-regular-expressions}
\end{figure}

We will look at a slightly more general question, which subsumes the
needs stated in the preceding paragraph. A \emph{generalized regular
  expression} (Figure~\ref{fig:generalized-regular-expressions}) is an
expression built from the regular operators empty set, empty word,
singleton word, alternative, concatenation, and Kleene star. In
addtion, it may contain the operators intersection and complement.


\subsection{Naive Approach}
\label{sec:naive-approach}




%%% Local Variables:
%%% mode: latex
%%% TeX-master: "main"
%%% End:
