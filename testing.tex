\section{Testing}
\label{sec:test}

We implemented our algorithms as libraries to create
test harnesses for regular expression implementations.
We used this library to implement a test harness for the \ocaml Re library%
\footnote{\url{https://github.com/ocaml/ocaml-re}},
a commonly used \ocaml regular expression implementation.
%
We also created a set of test cases for students projects in \haskell
which helped them write better implementations.

Both libraries provide test-harnesses which generate
regular expressions and positive and negative samples. The
implementation under test can then compile the regular expression and apply it
efficiently on the samples.
The library exposes the sample generation as a generator in the style of
property testing such as QuickCheck~\cite{DBLP:conf/icfp/ClaessenH00}.
This way we can use the tooling already available in such libraries.
%
The simplified API of the \ocaml version is shown below.
The main function \code{arbitrary n alphabet} returns a generator
which provides on average \code{n} samples using the given alphabet.

\begin{lstlisting}
type test = {
  re : Regex.t ;
  pos : Word.t list ;
  neg : Word.t list ;
}
val arbitrary:
  int -> Word.char list -> test QCheck.arbitrary
\end{lstlisting}

Regular expressions are easy to generate using QuickCheck-like
libraries as they are represented by an algebraic datatype.  The only
constraints we place on generated regular expressions is that
the star-height should be less than 3. While our technique can be used for
regular expressions with arbitrarily nested repetitions, it can cause
slowdown and large memory consumption which are
inconvenient in the context of automated testing.

Our testing library only returns a finite number of samples. However,
the language can (and often will) be infinite. We want to generate
test-cases that exercise the implementation under test as much as
possible. For this purpose, we use a technique similar to the fast
approximation for reservoir
sampling~\citep{DBLP:journals/toms/Vitter87}.  When considering the
sequence of words in the language, we skip $k$ elements where $k$
follows a power law of mean $n$. We then return the given sample, and
stop the sampling with a probability $1/n$.
%
Using this technique, we obtain on average $k$ samples that are regularly
spaced at the beginning of the stream, but will occasionally skip ahead
and return very large words. This approach has proven satisfactory at finding good
testing samples in practice.

%%% Local Variables:
%%% mode: latex
%%% TeX-master: "main"
%%% End:
