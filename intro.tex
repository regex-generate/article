\section{Introduction}

Regular languages are everywhere. Due to their apparent simplicity and
their concise representability in the form of regular expressions,
regular languages are used for many text processing
applications, reaching from text editors
\cite{DBLP:journals/cacm/Thompson68} to extracting data from web
pages.

Consequently, there a many algorithms and libraries that implement
parsing for regular expressions. Some of them are based on Thompson's
translation from regular expressions to nondeterministic finite
automata and then apply the powerset construction to obtain a
deterministic automaton. Others are based on Brzozowski's derivatives
\cite{Brzozowski1964} and
maps a regular expression directly to a deterministic
automaton. Antimirov's partial derivatives \cite{Antimirov96Partial}
yield another transformation into a nondeterministic automaton.
Yet others execute and memoize nondeterministic automata 

\begin{itemize}
\item Efficient testing and matching of deterministic regular
  expressions \cite{DBLP:journals/jcss/GrozM17}
\end{itemize}
\TODO{}

The implementation will be available on Github and will be submitted for artifact evaluation. 


We assume familiarity with Haskell throughout the paper.  Some
familiarity with formal languages is helpful, but not required as the
paper contains all relevant definitions. Our notation for formal
languages is borrowed from one of the classic textbooks on the topic
\cite{DBLP:books/daglib/0011126}.

%%% Local Variables:
%%% mode: latex
%%% TeX-master: "main"
%%% End:
